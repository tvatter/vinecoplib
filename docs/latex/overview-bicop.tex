Bivariate copula models are implemented as the {\ttfamily Bicop} class, and {\ttfamily Bicop\+Family} is a closely related enum class describing the type or \char`\"{}family\char`\"{} of copula. To use bivariate copula models in your code, include the header {\ttfamily \hyperlink{bicop_2class_8hpp_source}{vinecopulib/bicop/class.\+hpp}} (or simply {\ttfamily \hyperlink{vinecopulib_8hpp_source}{vinecopulib.\+hpp}}) at the top of your source file.\hypertarget{overview-bicop_Implemented}{}\section{bivariate copula families}\label{overview-bicop_Implemented}
\tabulinesep=1mm
\begin{longtabu} spread 0pt [c]{*{3}{|X[-1]}|}
\hline
\rowcolor{\tableheadbgcolor}\textbf{ type }&\textbf{ name }&\textbf{ Bicop\+Family  }\\\cline{1-3}
\endfirsthead
\hline
\endfoot
\hline
\rowcolor{\tableheadbgcolor}\textbf{ type }&\textbf{ name }&\textbf{ Bicop\+Family  }\\\cline{1-3}
\endhead
-\/ &Independence &indep \\\cline{1-3}
Elliptical &Gaussian &gaussian \\\cline{1-3}
&Student t &student \\\cline{1-3}
Archimedean &Clayton &clayton \\\cline{1-3}
&Gumbel &gumbel \\\cline{1-3}
&Frank &frank \\\cline{1-3}
&Joe &joe \\\cline{1-3}
&B\+B1 &bb1 \\\cline{1-3}
&B\+B6 &bb6 \\\cline{1-3}
&B\+B7 &bb7 \\\cline{1-3}
&B\+B8 &bb8 \\\cline{1-3}
Nonparametric &Transformation kernel &tll \\\cline{1-3}
\end{longtabu}
Note that several convenience vectors of families are included in the sub-\/namespace {\ttfamily bicop\+\_\+families}\+:
\begin{DoxyItemize}
\item {\ttfamily all} contains all the families
\item {\ttfamily parametric} contains the parametric families (all except {\ttfamily tll})
\item {\ttfamily nonparametric} contains the nonparametric families ({\ttfamily indep} and {\ttfamily tll})
\item {\ttfamily one\+\_\+par} contains the parametric families with a single parameter ({\ttfamily gaussian}, {\ttfamily clayton}, {\ttfamily gumbel}, {\ttfamily frank}, and {\ttfamily joe})
\item {\ttfamily two\+\_\+par} contains the parametric families with two parameters ({\ttfamily student}, {\ttfamily bb1}, {\ttfamily bb6}, {\ttfamily bb7}, and {\ttfamily bb8})
\item {\ttfamily elliptical} contains the elliptical families
\item {\ttfamily archimedean} contains the archimedean families
\item {\ttfamily bb} contains the BB families
\item {\ttfamily itau} families for which estimation by Kendall\textquotesingle{}s tau inversion is available ({\ttfamily indep},{\ttfamily gaussian}, {\ttfamily student},{\ttfamily clayton}, {\ttfamily gumbel}, {\ttfamily frank}, {\ttfamily joe})
\end{DoxyItemize}


\begin{DoxyCode}
\textcolor{comment}{// print all available families}
std::cout << \textcolor{stringliteral}{"Available families : "};
\textcolor{keywordflow}{for} (\textcolor{keyword}{auto} family : \hyperlink{namespacevinecopulib_1_1bicop__families_a5214a513f41ec23b74782aab96ea6774}{vinecopulib::bicop\_families::all}) \{
        std::cout << \hyperlink{namespacevinecopulib_ac46553ae5f99072f65e9d3254d2c526d}{get\_family\_name}(family) << \textcolor{stringliteral}{" "};
\}
\end{DoxyCode}
\hypertarget{overview-bicop_bicop-custom}{}\section{Set up a custom bivariate copula model}\label{overview-bicop_bicop-custom}
There are essentially two ways of setting-\/up bivariate copulas\+:
\begin{DoxyItemize}
\item with known parameters,
\item from data (i.\+e., with estimated parameters).
\end{DoxyItemize}

The constructor with known parameters takes 3 arguments\+:
\begin{DoxyItemize}
\item The copula family (default to {\ttfamily indep})
\item The rotation (default to {\ttfamily 0})
\item The parameters (default to parameters corresponding to an independence copula)
\end{DoxyItemize}


\begin{DoxyCode}
\textcolor{comment}{// 90 degree rotated Clayton with default parameter (corresponds to independence)}
Bicop \hyperlink{namespacevinecopulib_a42e95cc06d33896199caab0c11ad44f3ad0b8c94bb241f6f7a4b96cd6d3e26c36}{clayton}(BicopFamily::clayton, 90);

\textcolor{comment}{// Gauss copula with parameter 0.5}
Bicop gauss(BicopFamily::gaussian, 0,  Eigen::VectorXd::Constant(1, 0.5));
\end{DoxyCode}
 The constructor from data takes the same arguments as the select method and is described in the next section.\hypertarget{overview-bicop_bicop-fit}{}\section{Fit and select a bivariate copula}\label{overview-bicop_bicop-fit}
You can either fit the parameters of a given {\ttfamily Bicop} object with {\ttfamily fit()} or select the best fitting model from a set of families with {\ttfamily select()}.


\begin{DoxyCode}
create a Gauss copula with parameter 0.5 and simulate 1e3 observations
Bicop model(BicopFamily::gaussian, 0,  Eigen::VectorXd::Constant(1, 0.5));
\textcolor{keyword}{auto} data = model.simulate(1e3);

instantiate a \hyperlink{namespacevinecopulib_a42e95cc06d33896199caab0c11ad44f3a304e2a3b544f6b9f267a151e1bcee487}{gaussian} copula with \textcolor{keywordflow}{default} parameters and fit to data
Bicop fitted(BicopFamily::gaussian);
fitted.fit(data);
std::cout <<
       \textcolor{stringliteral}{"estimated parameter: "} <<
       fitted.get\_parameters() <<
       std::endl;

assign another family to the same variable and fit to data
fitted = Bicop(BicopFamily::student);
fitted.fit(data);
std::cout <<
       \textcolor{stringliteral}{"estimated parameter: "} <<
       fitted.get\_parameters() <<
       std::endl;

alternatively, assign to a family and fit automatically
fitted.select(data);
std::cout <<
       \textcolor{stringliteral}{"family: "} << fitted.get\_family\_name() <<
       \textcolor{stringliteral}{"rotation: "} <<  fitted.get\_rotation() <<
       std::endl;
\end{DoxyCode}


As it\textquotesingle{}s arguably the most important function of the {\ttfamily Bicop} class, it\textquotesingle{}s worth understanding the second argument of {\ttfamily select()}, namely an object of the class {\ttfamily Fit\+Controls\+Bicop}, which contain several data members\+:
\begin{DoxyItemize}
\item {\ttfamily std\+::vector$<$Bicop\+Family$>$ family\+\_\+set} describes the set of family to select from. It can take a user specified vector of families or any of those mentioned above (default is {\ttfamily bicop\+\_\+families\+::all}).
\item {\ttfamily std\+::string parametric\+\_\+method} describes the estimation method. It can take {\ttfamily \char`\"{}mle\char`\"{}} (default, for maximum-\/likelihood estimation) and {\ttfamily \char`\"{}itau\char`\"{}} (for Kendall\textquotesingle{}s tau inversion, although only available for families included in {\ttfamily bicop\+\_\+families\+::itau}).
\item {\ttfamily std\+::string nonparametric\+\_\+method} describes the degree of the density approximation for the transformation kernel estimator. It can take {\ttfamily constant}, {\ttfamily linear} and {\ttfamily quadratic} (default) for approximations of degree zero, one and two.
\item {\ttfamily double nonparametric\+\_\+mult} a factor with which the smoothing parameters are multiplied.
\item {\ttfamily std\+::string selection\+\_\+criterion} describes the criterion to compare the families. It can take either {\ttfamily \char`\"{}loglik\char`\"{}}, {\ttfamily \char`\"{}aic\char`\"{}}, or {\ttfamily \char`\"{}bic\char`\"{}}(default).
\item {\ttfamily Eigen\+::\+Vector\+Xd weights} an optional vector of weights for the observations.
\item {\ttfamily bool preselect\+\_\+families} describes a heuristic preselection method (default is {\ttfamily true}) based on symmetry properties of the data (e.\+g., the unrotated Clayton won\textquotesingle{}t be preselected if the data displays upper-\/tail dependence).
\item {\ttfamily size\+\_\+t num\+\_\+threads} number of threads to run in parallel when fitting several families.
\end{DoxyItemize}

As mentioned \href{#bicop-custom}{\tt above}, the arguments of {\ttfamily select()} can be used as arguments to a constructor allowing to instantiate a new object directly\+:


\begin{DoxyCode}
\textcolor{comment}{// instantiate an archimedean copula by selecting the "best" family according to}
\textcolor{comment}{// the BIC and parameters corresponding to the MLE}
Bicop best\_archimedean(data, FitControlsBicop(bicop\_families::archimedean));
std::cout <<
       \textcolor{stringliteral}{"family: "} << best\_archimedean.get\_family\_name() <<
       \textcolor{stringliteral}{"rotation: "} <<  best\_archimedean.get\_rotation() <<
       best\_archimedean.get\_parameters() <<
       std::endl

\textcolor{comment}{// instantiate a bivariate copula by selecting the "best" family according to}
\textcolor{comment}{// the AIC and parameters corresponding to Kendall's tau inversion}
FitControlsBicop controls(bicop\_families::itau, \textcolor{stringliteral}{"itau"});
controls.set\_selection\_criterion(\textcolor{stringliteral}{"aic"});
Bicop best\_itau(data, controls));
std::cout <<
       \textcolor{stringliteral}{"family: "} << best\_itau.get\_family\_name() <<
       \textcolor{stringliteral}{"rotation: "} <<  best\_itau.get\_rotation() <<
       best\_itau.get\_parameters() <<
       std::endl
\end{DoxyCode}
\hypertarget{overview-bicop_bicop-model}{}\section{Work with a bivariate copula model}\label{overview-bicop_bicop-model}
You can simulate from a model and evaluate the pdf, h-\/functions, inverse h-\/functions, log-\/likelihood, A\+IC, and B\+IC.


\begin{DoxyCode}
\textcolor{comment}{// Gauss copula with parameter 0.5}
Bicop bicop(BicopFamily::gaussian, 0,  Eigen::VectorXd::Constant(1, 0.5));

\textcolor{comment}{// Simulate 100 observations}
\textcolor{keyword}{auto} sim\_data = bicop.simulate(100);

\textcolor{comment}{// Evaluate the pdf}
\textcolor{keyword}{auto} pdf  = bicop.pdf(sim\_data);

\textcolor{comment}{// Evaluate the two h-functions}
\textcolor{keyword}{auto} h1   = bicop.hfunc1(sim\_data);
\textcolor{keyword}{auto} h2   = bicop.hfunc2(sim\_data);

\textcolor{comment}{// Evalute the two inverse h-functions}
\textcolor{keyword}{auto} hi1  = bicop.hinv1(sim\_data);
\textcolor{keyword}{auto} hi2  = bicop.hinv2(sim\_data);

\textcolor{comment}{// Evaluate the log-likelihood, AIC, and BIC}
\textcolor{keyword}{auto} ll   = bicop.loglik(sim\_data);
\textcolor{keyword}{auto} aic  = bicop.aic(sim\_data);
\textcolor{keyword}{auto} bic  = bicop.bic(sim\_data);
\end{DoxyCode}


Bivariate copula models can also be written to and constructed from J\+S\+ON files and {\ttfamily boost\+::property\+\_\+tree\+::ptree} objects\+:


\begin{DoxyCode}
\textcolor{comment}{// Gauss copula with parameter 0.5}
Bicop bicop(BicopFamily::gaussian, 0,  Eigen::VectorXd::Constant(1, 0.5));

\textcolor{comment}{// Save as a ptree object}
boost::property\_tree::ptree bicop\_node = bicop.to\_ptree();

\textcolor{comment}{// Write into a JSON file}
boost::property\_tree::write\_json(\textcolor{stringliteral}{"myfile.JSON"}, bicop\_node);

\textcolor{comment}{// Equivalently}
bicop.to\_json(\textcolor{stringliteral}{"myfile.JSON"});

\textcolor{comment}{// Then a new Bicop can be constructed from the ptree object}
Bicop bicop2(bicop\_node);

\textcolor{comment}{// Or from the JSON file}
Bicop bicop3(\textcolor{stringliteral}{"myfile.JSON"});
\end{DoxyCode}
 